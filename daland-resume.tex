%%%%%%%%%%%%%%%%%%%%%%%%%%%%%%%%%%%%%%%%%
% Twenty Seconds Resume/CV
% LaTeX Template
% Version 1.1 (8/1/17)
%
% This template has been downloaded from:
% http://www.LaTeXTemplates.com
%
% Original author:
% Carmine Spagnuolo (cspagnuolo@unisa.it) with major modifications by 
% Vel (vel@LaTeXTemplates.com)
%
% License:
% The MIT License (see included LICENSE file)
%
%%%%%%%%%%%%%%%%%%%%%%%%%%%%%%%%%%%%%%%%%

%----------------------------------------------------------------------------------------
%	PACKAGES AND OTHER DOCUMENT CONFIGURATIONS
%----------------------------------------------------------------------------------------

\documentclass[letterpaper]{twentysecondcv} % a4paper for A4

%----------------------------------------------------------------------------------------
%	 PERSONAL INFORMATION
%----------------------------------------------------------------------------------------

% If you don't need one or more of the below, just remove the content leaving the command, e.g. \cvnumberphone{}

\profilepic{logo.png} % Profile picture
% TODO: remove cvjobtitle and replace with picture that shows overlap of "Data" "Science" "Language" "Engineer" like in a word cloud or Venn diagram or something

\cvname{Robert Daland} % Your name
\cvjobtitle{} % Job title/career

\cvdate{} % Date of birth
\cvaddress{} % Short address/location, use \newline if more than 1 line is required
\cvnumberphone{+1 310-402-1176} % Phone number
\cvsite{\href{https://www.linkedin.com/in/robert-daland-176362111/}{LinkedIn homepage}} % Personal website
\cvmail{r.daland@gmail.com} % Email address

%----------------------------------------------------------------------------------------

\begin{document}

%----------------------------------------------------------------------------------------
%	 ABOUT ME
%----------------------------------------------------------------------------------------

\aboutme{} % To have no About Me section, just remove all the text and leave \aboutme{}

%----------------------------------------------------------------------------------------
%	 SKILLS
%----------------------------------------------------------------------------------------

% Skill bar section, each skill must have a value between 0 an 6 (float)
\skills{{R/3.0},{NumPy, SciPy, Jupyter, Pandas/3.0},{Graph theory/3.0},{Project management/4.0},{Corpus linguistics/4.0},{Statistics and quantitative analysis/4.0},{Language modeling/4.0},{Technical written communication/5.0},{Public speaking and presenting/5.0},{Research design and methodology/5.0},{Python/5.0},{Linguistic theory/6.0}}

%------------------------------------------------

% Skill text section, each skill must have a value between 0 an 6
\skillstext{}

%----------------------------------------------------------------------------------------

\makeprofile % Print the sidebar

%----------------------------------------------------------------------------------------
%	 INTERESTS
%----------------------------------------------------------------------------------------

\section{Interests}

Computer understanding of natural language (Automatic Speech Recognition, Natural Language Understanding).

Meaning of data: design, collection, analysis, visualization, exposition

%----------------------------------------------------------------------------------------
%	 EDUCATION
%----------------------------------------------------------------------------------------

\section{Education}

\begin{twenty} % Environment for a list with descriptions
	\twentyitem{2009}{PhD, Linguistics}{Specializing in Phonology; Minor in Cognitive Science}{Northwestern University}
	\twentyitem{2001}{MS, Mathematics}{Specializing in Iterated Function Systems}{NCSU (North Carolina State University)}
	\twentyitem{2001}{BA, English}{Specializing in literature; Minor in Religious Studies}{NCSU}
	\twentyitem{2000}{BS, Mathematics}{}{NCSU}
	%\twentyitem{<dates>}{<title>}{<location>}{<description>}
\end{twenty}

%----------------------------------------------------------------------------------------
%	 EXPERIENCE
%----------------------------------------------------------------------------------------

\section{Experience}

\begin{twenty} % Environment for a list with descriptions
	\twentyitem{12/2017 -- present \ \ \ \ \ \ }{Quality Engineer}{Natural Language Understanding \ \ \ \ \ \ \ \ \ \ \ \ }{Apple | Siri}
	\twentyitem{2009 -- 2017  }{Assistant Professor}{Phonology (Linguistics)\ \ \ \ \ \ \ \ \ \ \ \ }{UCLA}
	\twentyitem{2003 -- 2009  }{Graduate Student}{Linguistics\ \ \ \ \ \ \ \ \ \ \ \ }{Northwestern University}
	\twentyitem{summer 2001  }{Firmware Engineer}{Contractor\ \ \ \ \ \ \ \ \ \ \ \ }{Powerware/Invensys}
	%\twentyitem{<dates>}{<title>}{<location>}{<description>}
\end{twenty}

%----------------------------------------------------------------------------------------
%	 PUBLICATIONS
%----------------------------------------------------------------------------------------

\section{Selected Publications}

% 
\begin{twentyshort} % Environment for a short list with no descriptions
 	\twentyitemshort{accepted}{Mayer C \& Daland R. \href{https://doi.org/10.1162/ling_a_00359}{A method for projecting features from observed phonological classes.} \textit{Linguistic Inquiry}.}
         \twentyitemshort{2019}{Daland R, Oh M, \& Davidson L. \href{}{On the relation between speech perception and loanword adaptation.} \textit{Natural Language and Linguistic Theory 37}(3), 825-868.}
 	\twentyitemshort{2015}{Daland R. \href{https://doi.org/10.1017/S0952675715000251}{Long words in maximum entropy phonotactic grammars.} \textit{Phonology 32}(3), 353-383.}
 	\twentyitemshort{2015}{Norrmann I \& Daland R. \href{https://doi.org/10.1515/opli-2015-0024}{Phonetic evidence for the resyllabification account of vowel prothesis in Spanish speakers acquiring English [s]-consonant clusters.} \textit{Open Linguistics 1}(1).}
 	\twentyitemshort{2015}{Daland R, Oh M, \& Kim S. \href{https://doi.org/10.1016/j.lingua.2015.03.002}{When in doubt, read the instructions: Orthographic effect in loanword adaptation.} \textit{Lingua 159}, 70-92.}
 	\twentyitemshort{2014}{Daland R. \href{https://doi.org/10.3989/loquens.2014.004}{What is computational phonology?} (OPEN ACCESS) \textit{LOQUENS 1}(1), e400.}
 	\twentyitemshort{2013}{Cristia A, Daland R, Mielke J, \& Peperkamp S. \href{https://doi.org/10.1515/lp-2013-001}{Similarity in the generalization of implicitly learned sound patterns.} \textit{Laboratory Phonology 4}(2), 259-286.}
 	\twentyitemshort{2013}{Daland R \& Zuraw K. Does Korean defeat phonotactic word segmentation? \textit{ACL 51}, Sofia, Bulgaria, August 4-9, 2013.}
 	\twentyitemshort{2013}{Daland R. \href{https://doi.org/10.1017/S0305000912000372}{Variation in child-directed speech: A case study of manner class frequencies.} (OPEN ACCESS) \textit{Journal of Child Language 40}(5), 1091-1122.}
 	\twentyitemshort{2011}{Daland R, Hayes B, White J, Garellek M, Davis A, \& Norrmann I. \href{https://doi.org/10.1017/S0952675711000145}{Explaining sonority projection effects.} \textit{Phonology 28}(2), 197-234.}
 	\twentyitemshort{2011}{Daland R \& Pierrehumbert JB. \href{https://doi.org/10.1111/j.1551-6709.2010.01160.x}{Learning diphone-based segmentation.} \textit{Cognitive Science 35}(1), 119-155.}
 	\twentyitemshort{2009}{Goldrick M \& Daland R. \href{https://doi.org/10.1017/S0952675709001742}{Linking speech errors and phonological grammars: Insights from Harmonic Grammar networks.} \textit{Phonology 26}(1) (special issue on connecting theory and experimental methods), 147-185.}
 	\twentyitemshort{2007}{Daland R, Sims AD, \& Pierrehumbert JB. Much ado about nothing: A social network model of Russian paradigmatic gaps. \textit{ACL 45}, Prague, June 23rd-June 30th.}
	%\twentyitem{<dates>}{<title>}{<location>}{<description>}
\end{twentyshort}

	
%----------------------------------------------------------------------------------------
%	 OTHER INFORMATION
%----------------------------------------------------------------------------------------


%----------------------------------------------------------------------------------------
%	 SECOND PAGE EXAMPLE
%----------------------------------------------------------------------------------------

%\newpage % Start a new page

%\makeprofile % Print the sidebar

%\section{Other information}

%\subsection{Review}

%Alice approaches Wonderland as an anthropologist, but maintains a strong sense of noblesse oblige that comes with her class status. She has confidence in her social position, education, and the Victorian virtue of good manners. Alice has a feeling of entitlement, particularly when comparing herself to Mabel, whom she declares has a ``poky little house," and no toys. Additionally, she flaunts her limited information base with anyone who will listen and becomes increasingly obsessed with the importance of good manners as she deals with the rude creatures of Wonderland. Alice maintains a superior attitude and behaves with solicitous indulgence toward those she believes are less privileged.

%\section{Other information}

%\subsection{Review}

%Alice approaches Wonderland as an anthropologist, but maintains a strong sense of noblesse oblige that comes with her class status. She has confidence in her social position, education, and the Victorian virtue of good manners. Alice has a feeling of entitlement, particularly when comparing herself to Mabel, whom she declares has a ``poky little house," and no toys. Additionally, she flaunts her limited information base with anyone who will listen and becomes increasingly obsessed with the importance of good manners as she deals with the rude creatures of Wonderland. Alice maintains a superior attitude and behaves with solicitous indulgence toward those she believes are less privileged.

%----------------------------------------------------------------------------------------

\end{document} 
